\documentclass[a4paper]{book}
\usepackage[ae,hyper]{Rd}
\usepackage{makeidx}
\makeindex{}
\begin{document}
\chapter*{}
\begin{center}
{\textbf{\huge \R{ documentation}} \par\bigskip{{\Large of \file{Rd2.dvi} etc.}}}
\par\bigskip{\large \today}
\end{center}
\Rdcontents{\R{} topics documented:}
\Header{}{}

\Header{aldh2}{ALDH2 markers and Alcoholism}
\keyword{datasets}{aldh2}
\begin{Description}\relax
This data set contains eight ALDH2 markers 
and Japanese patients and controls.\end{Description}
\begin{Usage}
\begin{verbatim}data(aldh2)\end{verbatim}
\end{Usage}
\begin{Format}\relax
A data frame:
\describe{
\item[id] subject id
\item[y] a variable taking value 0 for controls and 1 for Schizophrenia
\item[D12S2070.a1] D12S2070 allele a1
\item[D12S2070.a2] D12S2070 allele a2
\item[D12S839.a1] D12S839 allele a1
\item[D12S839.a2] D12S839 allele a2
\item[D12S821.a1] D12S821 allele a1
\item[D12S821.a2] D12S821 allele a2
\item[D12S1344.a1] D12S1344 allele a1
\item[D12S1344.a2] D12S1344 allele a2
\item[EXON12.a1] EXON12 allele a1
\item[EXON12.a2] EXON12 allele a2
\item[EXON1.a1] EXON1 allele a1
\item[EXON1.a2] EXON1 allele a2
\item[D12S2263.a1] D12S2263 allele a1
\item[D12S2263.a2] D12S2263 allele a2
\item[D12S1341.a1] D12S1341 allele a1
\item[D12S1341.a2] D12S1341 allele a2
}

The remaining variables are genotypes for 8 loci, with a prefix name
(e.g., "EXON12") and a suffix for each of two alleles (".a1" and ".a2").

The eight markers loci follows the following map (base pairs)

\Tabular{lr}{
D12S2070   & (> 450 000),\\
D12S839    & (> 450 000),\\
D12S821    & (\eqn{\sim}{~} 400 000),\\
D12S1344   & (   83 853),\\
EXON12     & (    0),\\
EXON1      & (   37 335),\\
D12S2263   & (   38 927),\\
D12S1341   & (> 450 000)
}\end{Format}
\begin{Source}\relax
Prof Ian Craig of Oxford and SGDP Centre, KCL\end{Source}

\Header{apoeapoc}{APOE/APOC1 markers and Schizophrenia}
\keyword{datasets}{apoeapoc}
\begin{Description}\relax
This data set contains APOE/APOC1 markers 
and Chinese Schizophrenic patients and controls.\end{Description}
\begin{Usage}
\begin{verbatim}data(apoeapoc)\end{verbatim}
\end{Usage}
\begin{Format}\relax
A data frame:
\describe{
\item[id] subject id
\item[y] a variable taking value 0 for controls and 1 for Schizophrenia
\item[sex] sex
\item[age] age
\item[apoe.a1] APOE allele a1
\item[apoe.a2] APOE allele a2
\item[apoc.a1] APOC allele a1
\item[apoc.a2] APOC allele a2
}

The remaining variables are age, sex and genotypes for APOE and APOC 
with suffixes for each of two alleles (".a1" and ".a2").\end{Format}
\begin{Source}\relax
Dr JJ Shi of Western China Medical University\end{Source}

\Header{bt}{Bradley-Terry model for contingency table}
\begin{Description}\relax
This function calculates statistics under Bradley-Terry model. 

Inside the function is a function toETDT which generates data required by ETDT.\end{Description}
\begin{Usage}
\begin{verbatim}bt(x)\end{verbatim}
\end{Usage}
\begin{Arguments}
\begin{ldescription}
\item[\code{x}] the data table
\end{ldescription}
\end{Arguments}
\begin{Value}
The returned value is a list containing:
\begin{ldescription}
\item[\code{y}] A column of 1
\item[\code{count}] the frequency count/weight
\item[\code{allele}] the design matrix
\item[\code{bt.glm}] a glm.fit object
\item[\code{etdt.dat}] a data table that can be used by ETDT
\end{ldescription}
\end{Value}
\begin{Section}{References}
Bradley RA, Terry M E (1952) Rank analysis of incomplete block designs I. 
the method of paired comparisons. Biometrika 39:324--345

Sham PC, Curtis D (1995) An extended transmission/disequilibrium 
test ({TDT}) for multi-allelic marker loci. Ann. Hum. Genet. 59:323-336\end{Section}
\begin{Author}\relax
Jing hua Zhao\end{Author}
\begin{SeeAlso}\relax
\code{\Link{mtdt}}\end{SeeAlso}
\begin{Examples}
\begin{ExampleCode}

# Copeman JB, Cucca F, Hearne CM, Cornall RJ, Reed PW, Ronningen KS, Undlien DE, Nistico L, Buzzetti R, Tosi R, et al.
# (1995) Linkage disequilibrium mapping of a type 1 diabetes susceptibility gene (IDDM7) to chromosome 2q31-q33. 
# Nat Genet 9: 80-5

x <- matrix(c(0,0, 0, 2, 0,0, 0, 0, 0, 0, 0, 0,
              0,0, 1, 3, 0,0, 0, 2, 3, 0, 0, 0,
              2,3,26,35, 7,0, 2,10,11, 3, 4, 1,
              2,3,22,26, 6,2, 4, 4,10, 2, 2, 0,
              0,1, 7,10, 2,0, 0, 2, 2, 1, 1, 0,
              0,0, 1, 4, 0,1, 0, 1, 0, 0, 0, 0,
              0,2, 5, 4, 1,1, 0, 0, 0, 2, 0, 0,
              0,0, 2, 6, 1,0, 2, 0, 2, 0, 0, 0,
              0,3, 6,19, 6,0, 0, 2, 5, 3, 0, 0,
              0,0, 3, 1, 1,0, 0, 0, 1, 0, 0, 0,
              0,0, 0, 2, 0,0, 0, 0, 0, 0, 0, 0,
              0,0, 1, 0, 0,0, 0, 0, 0, 0, 0, 0),nrow=12)

# Bradley-Terry model, only deviance is available in R
bt.ex<-bt(x)
anova(bt.ex$bt.glm)
summary(bt.ex$bt.glm)
bt.ex$etdt.dat

\end{ExampleCode}
\end{Examples}

\Header{chow.test}{Chow's test for heterogeneity in two regressions}
\begin{Description}\relax
Chow's test is for differences between two or more regressions.  Assuming that
errors in regressions 1 and 2 are normally distributed with zero mean and
homoscedastic variance, and they are independent of each other, the test of
regressions from sample sizes \eqn{n_1}{} and \eqn{n_2}{} is then carried out using
the following steps.  1.  Run a regression on the combined sample with size
\eqn{n=n_1+n_2}{} and obtain within group sum of squares called \eqn{S_1}{}.  The
number of degrees of freedom is \eqn{n_1+n_2-k}{}, with \eqn{k}{} being the number
of parameters estimated, including the intercept.  2.  Run two regressions on
the two individual samples with sizes \eqn{n_1}{} and \eqn{n_2}{}, and obtain their
within group sums of square \eqn{S_2+S_3}{}, with \eqn{n_1+n_2-2k}{} degrees of
freedom.  3.  Conduct an \eqn{F_{(k,n_1+n_2-2k)}}{} test defined by \deqn{F =
\frac{[S_1-(S_2+S_3)]/k}{[(S_2+S_3)/(n_1+n_2-2k)]}}{} If the \eqn{F}{} statistic
exceeds the critical \eqn{F}{}, we reject the null hypothesis that the two
regressions are equal.

In the case of haplotype trend regression, haplotype frequencies from combined
data are known, so can be directly used.\end{Description}
\begin{Usage}
\begin{verbatim}chow.test(y1,x1,y2,x2,x=NULL)\end{verbatim}
\end{Usage}
\begin{Arguments}
\begin{ldescription}
\item[\code{y1}] a vector of dependent variable
\item[\code{x1}] a matrix of independent variables
\item[\code{y2}] a vector of dependent variable
\item[\code{x2}] a matrix of independent variables
\item[\code{x}] a known matrix of independent variables
\end{ldescription}
\end{Arguments}
\begin{Value}
The returned value is a vector containing (please use subscript to access them):

\begin{ldescription}
\item[\code{F}] the F statistic
\item[\code{df1}] the numerator degree(s) of freedom
\item[\code{df2}] the denominator degree(s) of freedom
\item[\code{p}] the p value for the F test
\end{ldescription}
\end{Value}
\begin{Section}{References}
Chow GC (1960). Tests of equality between sets of coefficients in two linear regression. Econometrica 28:591-605\end{Section}
\begin{Note}\relax
adapted from chow.R\end{Note}
\begin{Author}\relax
Shigenobu Aoki, Jing hua Zhao\end{Author}
\begin{Source}\relax
\url{http://aoki2.si.gunma-u.ac.jp/R/}\end{Source}
\begin{SeeAlso}\relax
\code{\Link{htr}}\end{SeeAlso}
\begin{Examples}
\begin{ExampleCode}

dat1 <- matrix(c(
        1.2, 1.9, 0.9,
        1.6, 2.7, 1.3,
        3.5, 3.7, 2.0,
        4.0, 3.1, 1.8,
        5.6, 3.5, 2.2,
        5.7, 7.5, 3.5,
        6.7, 1.2, 1.9,
        7.5, 3.7, 2.7,
        8.5, 0.6, 2.1,
        9.7, 5.1, 3.6, byrow=TRUE, ncol=3)

dat2 <- matrix(c(
        1.4, 1.3, 0.5,
        1.5, 2.3, 1.3,
        3.1, 3.2, 2.5,
        4.4, 3.6, 1.1,
        5.1, 3.1, 2.8,
        5.2, 7.3, 3.3,
        6.5, 1.5, 1.3,
        7.8, 3.2, 2.2,
        8.1, 0.1, 2.8,
        9.5, 5.6, 3.9), byrow=TRUE, ncol=3)

y1<-dat1[,3]
y2<-dat2[,3]
x1<-dat1[,1:2]
x2<-dat2[,1:2]
chow.test.r<-chow.test(y1,x1,y2,x2)

\end{ExampleCode}
\end{Examples}

\Header{fbsize}{Sample size for family-based linkage and association design}
\begin{Description}\relax
This function implements Risch and Merikangas (1996) statistics 
evaluating power for family-based linkage and association design.
They are potentially useful in the prospect of genome-wide 
association studies.

The function calls auxiliary functions sn() and strlen; sn() 
contains the necessary thresholds for power calculation while
strlen() evaluates length of a string (generic).\end{Description}
\begin{Usage}
\begin{verbatim}fbsize(gamma,p,debug=0,error=0)\end{verbatim}
\end{Usage}
\begin{Arguments}
\begin{ldescription}
\item[\code{gamma}] genotype relative risk assuming multiplicative model
\item[\code{p}] frequency of disease allele
\item[\code{debug}] verbose output
\item[\code{error}] 0=use the correct formula,1=the original paper
\end{ldescription}
\end{Arguments}
\begin{Value}
The returned value is a list containing:

\begin{ldescription}
\item[\code{gamma}] input gamma
\item[\code{p}] input p
\item[\code{n1}] sample size for ASP
\item[\code{n2}] sample size for TDT
\item[\code{n3}] sample size for ASP-TDT
\item[\code{lambdao}] lambda o
\item[\code{lambdas}] lambda s
\end{ldescription}
\end{Value}
\begin{Section}{References}
Risch, N. and K. Merikangas (1996). The future of genetic studies of
complex human diseases. Science 273(September): 1516-1517.

Risch, N. and K. Merikangas (1997). Reply to Scott el al. Science
275(February): 1329-1330.

Scott, W. K., M. A. Pericak-Vance, et al. (1997). Genetic analysis of 
complex diseases. Science 275: 1327.\end{Section}
\begin{Note}\relax
extracted from rm.c\end{Note}
\begin{Author}\relax
Jing hua Zhao\end{Author}
\begin{SeeAlso}\relax
\code{\Link{pbsize}}\end{SeeAlso}
\begin{Examples}
\begin{ExampleCode}

models <- matrix(c(
    4.0, 0.01,
    4.0, 0.10,
    4.0, 0.50, 
    4.0, 0.80,
    2.0, 0.01,
    2.0, 0.10,
    2.0, 0.50,
    2.0, 0.80,
    1.5, 0.01,    
    1.5, 0.10,
    1.5, 0.50,
    1.5, 0.80), ncol=2, byrow=TRUE)
    
cat("\nThe family-based result: \n")
cat("\ngamma   p     Y     N_asp   P_A    Het    N_tdt  Het N_asp/tdt  L_o  L_s\n\n")
for(i in 1:12) {
  g <- models[i,1]
  p <- models[i,2]
  fbsize(g,p)
  if(i%%4==0) cat("\n")
}

# APOE-4, Scott WK, Pericak-Vance, MA & Haines JL
# Genetic analysis of complex diseases 1327
g <- 4.5
p <- 0.15
cat("\nAlzheimer's:\n\n")
fbsize(g,p)

\end{ExampleCode}
\end{Examples}

\Header{fsnps}{A case-control data involving four SNPs for missing genotype}
\keyword{datasets}{fsnps}
\begin{Description}\relax
This is a simulated data\end{Description}
\begin{Usage}
\begin{verbatim}data(hla)\end{verbatim}
\end{Usage}
\begin{Format}\relax
A data frame
\describe{
\item[id] subject id
\item[y] a column of 0/1 indicating case/control
\item[site1.a1] SNP 1 allele a1
\item[site1.a2] SNP 1 allele a2
\item[site2.a1] SNP 2 allele a1
\item[site2.a2] SNP 2 allele a2
\item[site3.a1] SNP 3 allele a1
\item[site3.a2] SNP 3 allele a2
\item[site4.a1] SNP 4 allele a1
\item[site4.a2] SNP 4 allele a2
}

The remaining variables are genotypes for 4 SNPs, coded in characters\end{Format}
\begin{Source}\relax
Dr Sebastien Lissarrague of Genset\end{Source}

