% \VignetteIndexEntry{An overview of gap}
% \Vignettekeywords{human genetics, linkage analysis, association analysis}
% \VignettePackage{gap}


\documentclass[10pt,a4paper]{article}

%\usepackage{a4wide}
%\usepackage{amsmath,epsfig,psfig}
%\usepackage{graphicx}
%\usepackage[authoryear,round]{natbib}

% \newcommand{\keywords}[1]{\addvspace{\baselineskip}\noindent{{\bf Keywords.} #1}}

\parindent 0in

\usepackage{/people/jzhao/R-alpha/share/texmf/Sweave}
\begin{document}
\title{An Integrated Genetic Analysis Package Using R}

\author{Jing hua Zhao}
\date{}
\maketitle

\begin{center}
Department of Epidemiology and Public Health, Unversity College London\\
http://www.ucl.ac.uk/$\sim$rmjdjhz, http://www.hgmp.mrc.ac.uk/$\sim$jzhao
\end{center}

\tableofcontents

% library(tools)
% Rnwfile<- file.path("/people/jzhao/cran/gap/inst/doc","gap.Rnw")
% Sweave(Rnwfile,pdf=TRUE,eps=TRUE,stylepath=TRUE,driver=RweaveLatex())

%%%%%%%%%%%%%%%%%%%%%%%%%%%%%%%%%%%%%%%%%%%%%%%%%%%%%%%%%%%%%%%%%%%%%%%%%%%

\section{Introduction}

This package was designed to integrate some C/Fortran/SAS programs I have written 
or used over the years. As such, it would rather be a long-term project, but an
immediate benefit would be something complementary to other packages currently
available in R, e.g. {\bf genetics}, {\bf hwde}, {\bf haplo.score}, etc. I hope
eventually this will be part of a bigger effort to fulfill most of the requirements
foreseen by many, e.g. Guo and Lange (2000), within the portable environment of R
for data management, analysis, graphics and object-oriented programming. \\

So far the number of functions is quite limited and experimental, but I already
feel enormous advantage by shifting to R and would like sooner rather than later
to share my work with others. I will not claim this work is exclusively done by
me, but would like to invite others to join me and enlarge the collections and
improve them.


\section{Implementation}

The following, extracted from the package INDEX, shows the data and functions
currently available.

\begin{verbatim}
aldh2                   ALDH2 markers and Alcoholism
apoeapoc                APOE/APOC1 markers and Schizophrenia
bt                      Bradley-Terry model for contingency table
chow.test               Chow's test for heterogeneity in two
                        regressions
cf                      Cystic Fibrosis data
crohn                   Crohn disease data
fa                      Friedreich Ataxia data
fbsize                  Sample size for family-based linkage and
                        association design
fsnps                   A case-control data involving four SNPs with
                        missing genotype
gc.em                   Gene counting for haplotype analysis
gcontrol                genomic control
gcp                     Permutation tests using GENECOUNTING
genecounting            Gene counting for haplotype analysis
gif                     Kinship coefficient and genetic index of
                        familiality
hap                     Haplotype reconstruction
hap.em                  Gene counting for haplotype analysis
hap.score               Score Statistics for Association of Traits
                        with Haplotypes
hla                     HLA markers and Schizophrenia
htr                     Haplotype trend regression
hwe                     Hardy-Weinberg equlibrium test for
                        multiallelic marker
hwe.hardy               Hardy-Weinberg equlibrium test using MCMC
kbyl                    LD statistics for two multiallelic loci
kin.morgan              kinship matrix for simple pedigree
makeped                 A function to prepare pedigrees in
                        post-MAKEPED format
mao                     A study of Parkinson's disease and MAO gene
mia                     multiple imputation analysis for hap
mtdt                    Transmission/disequilibrium test of a
                        multiallelic marker
muvar                   Means and variances under 1- and 2- locus
                        (biallelic) QTL model
nep499                  A study of Alzheimer's disease with eight SNPs and APOE
pbsize                  Power for population-based association design
pedtodot                Converting pedigree(s) to dot file(s)
pfc                     Probability of familial clustering of disease
pfc.sim                 Probability of familial clustering of disease
pgc                     Preparing weight for GENECOUNTING
plot.hap.score          Plot Haplotype Frequencies versus Haplotype
                        Score Statistics
print.hap.score         Print a hap.score object
s2k                     Statistics for 2 by K table
snca                    A study of Parkinson's disease and SNCA makers
tbyt                    LD statistics for two SNPs
whscore                 Whittemore-Halpern scores for allele-sharing
\end{verbatim}

Assuming proper installation, you will be able to obtain the list by typing
\textit{library(help=gap)} or view the list within a web browser via 
\textit{help.start()}.\\

You can cut and paste examples at end of each function's documentation.\\

Both \textit{genecounting} and \textit{hap} are able to handle SNPs and multiallelic
markers, with the former be flexible enough to include features such as X-linked data
(not incorporated yet) and the later being able to handle large number of SNPs, an
advantage over algorithms in {\bf haplo.score}. But the latter is able to recode
allele labels automatically, so functions \textit{gc.em} and \textit{hap.em}
are in {\bf haplo.score}'s \textit{haplo.em} format and used by a modified function
\textit{hap.score} in association testing.\\

It is notable that multilocus data are handled differently from that in {\bf hwde} and
elegant definitions of basic genetic data can be found in {\bf genetics} package.\\

Incidentally, I found my mixed-radixed sorting routine in C (Zhao \& Sham 2003) is much
faster than R's internal function.\\

With exceptions such as function \textit{pfc} which is very computer-intensive, most
functions in the package can easily be adapted for analysis of large datasets involving
either SNPs or multiallelic markers. Some are utility functions, e.g. \textit{muvar}
and \textit{whscore}, which will be part of the other analysis routines in the future.\\

For users, all functions have unified format. For developers, it is able to incorporate
their C/C++ programs more easily and avoid repetitive work such as preparing own routines
for matrix algebra and linear models. Further advantage can be taken from packages in
{\bf Bioconductor}, which are designed and written to deal with large number of genes.


\section{Examples}

Examples can be found from most function documentations. You can also try several simple
examples via \textit{demo}:

\begin{Schunk}
\begin{Sinput}
> library(gap)
> demo(gap)
\end{Sinput}
\end{Schunk}

\section{Known bugs}

Unaware of any bug after hwe.hardy was fixed. However, better memory management is expected.


\section{References}
\begin{itemize}
\item[] 
Chow GC (1960). Tests of equality between sets of coefficients in two
linear regression. Econometrica 28:591-605

\item[] 
Devlin B, Roeder K (1999) Genomic control for association studies. 
Biometrics 55:997-1004

\item[] 
Gholamic K, Thomas A (1994) A linear time algorithm for calculation of
multiple pairwise kinship coefficients and genetic index of familiality.
Comp Biomed Res 27:342-350

\item[] 
Guo S-W, Thompson EA (1992) Performing the Exact Test of
Hardy-Weinberg Proportion for Multiple Alleles. Biometrics. 48:361--372.

\item[] 
Guo S-W, Lange K (2000) Genetic mapping of complex traits: promises, problems, 
and prospects. Theor Popul Biol 57:1-11

\item[] 
Hirotsu C, Aoki S, Inada T, Kitao Y (2001) An exact test for the association
between the disease and alleles at highly polymorphic loci with particular interest
in the haplotype analysis. Biometrics 57:769-778

\item[] 
Miller MB (1997) Genomic scanning and the transmission/disequilibrium test:
analysis of error rates. Genet Epidemiol 14:851-856

\item[] 
Risch N, Merikangas K (1996). The future of genetic studies of
complex human diseases. Science 273(September): 1516-1517.
 
\item[] 
Risch N, Merikangas K (1997). Reply to Scott el al. Science
275(February): 1329-1330. 

\item[] 
Sham PC (1997) Transmission/disequilibrium tests for multiallelic loci.
Am J Hum Genet 61:774-778

\item[] 
Sham PC (1998). Statistics in Human Genetics. Arnold

\item[] 
Spielman RS, Ewens WJ (1996) The TDT and other family-based tests for
linkage disequilibrium and association. Am J Hum Genet 59:983-989

\item[] 
Zapata C, Carollo C, Rodriquez S (2001) Sampleing variance and distribution
of the D' measure of overall gametic disequilibrium between multiallelic loci.
Ann Hum Genet 65: 395-406

\item[] 
Zaykin DV, Westfall PH, Young SS, Karnoub MA, Wagner MJ, Ehm MG (2002)
Testing association of statistically inferred haplotypes with discrete
and continuous traits in samples of unrelated individuals. Hum
Hered 53:79-91

\item[] 
Zhao JH, Lissarrague S, Essioux L, Sham PC (2002).
GENECOUNTING: haplotype analysis with missing genotypes.
Bioinformatics 18(12):1694-1695
 
\item[] 
Zhao JH, Sham PC, Curtis D (1999) A program for the Monte Carlo evaluation
of significance of the extended transmission/disequilibrium test.
Am J Hum Genet 64:1484-1485

\item[] 
Zhao JH, Sham PC (2003). Generic number systems and haplotype
analysis. Comp Meth Prog Biomed 70: 1-9

Zhao JH (2004). 2LD, GENECOUNTING and HAP: Computer programs for linkage
disequilibrium analysis. Bioinformatics, 20, 1325-1326 

\end{itemize}

\end{document}
