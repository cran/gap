\documentclass[12pt]{article}
\begin{document}
\centerline{\large SIM -- a simulation program for genetic linkage and association}

\bigskip

{\bf SIM} is a program to simulate multiple traits each with both genetic and environmental
effects to be specified. The genetic components include 
major gene effects as well as polygenic effects and the genetic loci can be
either in linkage equilibrium or in linkage disequilibrium. It uses input formats
similar to LINKAGE files and outputs appropriate pedigrees from the simulation.

\section{The genetic model}

This program implements the genetic model $$x=g+p+c+e$$ where $x$ is a multivariate trait
($x_i,i=1,...,NTRAIT$), $g$ represents major loci ($g_j,j=1,...,NMG$), $p$ represents
polygenic effects ($p_k,k=1,...,NPG)$, $c$ is the common environment ($c_l,l=1,...,NCE$) and
$e$ represents unique environments ($e_m,m=1,...,NUE$). The variables $g, p, c, e$ are
referred to collectively as causal components of the trait. The $NMG$ major loci are a subset
of $NLOCI$ loci for which the order and recombination fractions are
($\theta_1,\theta_2,...,\theta_{NLOCI-1}$), and the number of alleles and allele frequencies
are specified. The meaning of some model constants are summarized in table~\ref{tab1}.

\begin{table}[h]\centering
\caption{Some model constants\label{tab1}}
\begin{tabular}{l|l}
\multicolumn{2}{c}{}\\
\hline
Parameter & Meaning \\
\hline
$MAXLOCI$ & maximum \# of loci\\
$MAXALLELES$ & maximum \# of alleles at each locus\\
$MAXFAM$ & maximum \# of families\\
$MAXIND$ & maximum \# of individuals within a family\\
\\
$NTRAIT$ & \# of traits\\
$NLOCI$  & \# of loci\\
$NMG$ & \# of major genes\\
$NPG$ & \# of polygenes\\
$NCE$ & \# of common environments\\
$NUE$ & \# of unique environments\\
$NDISEQ$ & \# of pairs in disequilibrium\\
$\beta$ & matrix of regression coefficients for each trait\\
\hline
\end{tabular}\end{table}

Major locus effects AA, Aa, aa are characterized by $z,q,t,d$, where $z$ is the mean effect
of AA, $q$ is the allele frequency of a, $t$ is the displacement between AA and aa and $d$ is the
dominance (see table~\ref{tab2}).

\begin{table}[h]\centering
\caption{Major locus parameters\label{tab2}}
\begin{tabular}{lll}
\multicolumn{3}{c}{}\\
\hline
Genotype & Frequencies &Effect(g) \\
\hline
AA & $p^2$ &$z$   \\
Aa & $2pq$ &$z+dt$\\
aa & $q^2$ &$z+t$ \\
\hline
\end{tabular}
\end{table}

Suppose $g$ has mean 0 and variance 1, we have $$z=-(tq^2+2pqdt)$$ and
$$t^2=\frac{1}{(pq)^2(q+2pd)^2+2pq(d-q^2-2pqd)^2+q^2(1-q^2-2pqd)^2}$$
so that the only free parameters are $q$ and $d$.

The polygenic effect of an offspring, conditional on those of parents ($p_F,p_M$), is
$$p_o={{{p_F+p_M}\over{2}}+{u\over\sqrt{2}}}$$ $u\sim N(0,1)$ is specific for each offspring.

Suppose the degree of transmission of parental shared environment to offspring is $k$, then
under no assortative mating, the common environment for offspring is $k
(c_F+c_M)+v\sqrt{(1-2k^2)}$, where $v\sim N(0,1)$ is the same within the whole sibship.  The
valid range for $k$ is $(0,1/\sqrt{2})$.

The unique environment is different among individuals and is $N(0,1)$.

All latent variables ($g,p,c,e$) have mean 0 and variance 1.

The regression equation of trait $x$ on latent variables is
\begin{eqnarray}
x_i&=&\beta_{ig_1}g_1+\cdots+\beta_{ig_{NMG}}g_{NMG} \cr
   &+&\beta_{ip_1}p_1+\cdots+\beta_{ip_{NPG}}p_{NPG} \cr
   &+&\beta_{ic_1}c_1+\cdots+\beta_{ic_{NCE}}c_{NCE} \cr
   &+&\beta_{ie_1}e_1+\cdots+\beta_{ie_{NUE}}e_{NUE}
\end{eqnarray}

The mean and variance of $x_i$ are then 0 and $\sum \beta_{ij}^2, j=1, ..., NMG + NPG + NCE +
NUE$. To ease the specifications, the input $\beta$'s are standardized by this in order to
have the unit trait variance.

\subsection{Introduction of linkage disequilibrium}

It is possible to include disequilibrium between multiallelic markers. Without loss of
generality, consider two marker loci, with $m$ and $n$ alleles and allele frequencies
$p_1,p_2,...,p_m$ and $q_1,q_2,...,q_n$, then we can arrange the haplotype
frequencies into a $m\times n$ contingency table. 

The $m\times n$ contingency table $\chi^2$-squared statistic $$\chi^2=N(
\sum_{i=1}^m\sum_{j=1}^n \frac{h_{ij}^2}{p_iq_i}-1)$$ where $N$ is the total cell counts. 

Under Hardy-Weinberg equilibrium, the haplotype frequencies can be obtained from the
marginal allele frequencies. The above expression can be rewritten as a familar quadratic
form. let ${\bf w}$ be the $m\times n$ diagonal matrix of direct product of the original
marginals, the $\chi^2$-squared statistic is ${\bf h}{\bf w}^{-1}{\bf h}'-1$, where
{\bf h} contains the actual haplotype frequencies.

Disequilibria among different alleles are introduced by numerical
optimization of the $\chi^2$-sqaured statistic over nonnegative
contraints of haplotype frequencies to get their
estimates. The constraints are formed through an iterative procedure to
allow for individual disequilibrium specifications. 

If $m,n=2$, this will reduce to the familiar case of $2\times 2$ table.


\section{The implementation}

The program first gets parental haplotypes from population frequencies and then proceeds
to generate the genotypes for children. Some loci are assigned as major
locus with dominance parameters. The disequilibrium is simulated if the disequilibrium
matrix of haplotype frequencies is specified.

The usual way of obtaining allele in a specific locus is as described in Weir (1990). 
Consider a 4-allele locus with allele frequencies $p_1, p_2, p_3, p_4$, when drawing a
(pseudo)random number from [0,1] we could divide the unit interval into four segements,
with boundaries $0, p_1, p_1+p_2, p_1+p_2+p_3, 1$, these segements therefore have lengths
$p_1, p_2, p_3, p_4$.  We assign allele number $i$ if the random number falls within the
$i$th interval.  Similar logic is applicable to simulation of recombination events. We can
achieve this by using other algorithms for generating random variable from multinomial
distribution.

The program has a setup that would be suitable for taking information from pedigree files
similar to LINKAGE and SIMULATE. Figure~\ref{fig1} depicts the steps.

\begin{figure}[h]
\caption{Flowchart of SIM\label{fig1}}
\begin{tabular}{llc}
\\
&&\framebox{simulate/read pedigree}\\
&&$\downarrow$\\
\framebox{model parameters}&$\longrightarrow$&\framebox{SIM}\\
&&$\downarrow$\\
&&\framebox{pedgree file}\\
&&$\downarrow$\\
&&\framebox{selection}\\
&&$\downarrow$\\
&&\framebox{final pedigree file}
\end{tabular}
\end{figure}

Currently the step \framebox{SIM} processes one family at a time, so that certain
selection procedure can be adopted easily.  A separate SAS program was written to generate
the disequilibrium coefficients from linearly constrained optimization. To account for
possible extensions involving many other statistical distributions, the freeware RANLIB
was used. Finally, to keep the program in its pure C form, the Numerical Recipes utitlity
nrutil.h/nrutil.c has been used to create dynamic arrays and matrices, although this could
be easily achieved with the {\bf new/delete} construct in C++.

A compiling control file Makefile has been created and tested under Sun Solaris and DEC 
Alpha with GNU C++ compiler, so that a simple command {\bf make} would generate the 
executable. It has also been tested under PC with Symantec C++7.2. Prolix output could be 
obtained by specifying \#undef DEBUG statement in include/sim.h.

The format of the command would be as follows: 

\bigskip
SIM locusfile pedigreefile controlfile outputfile 
\bigskip

A simple example is provided here. The locus file, pedigree file, control file, and
output file are loc.tst, ped.tst, problem.dat and sim.out, respectively.

\bigskip\noindent{\bf loc.tst}

\begin{small}\begin{verbatim}
3 0 0 0 1 0 2 2 2 0 <<NLOCI,riskloci,SEXLINK,program,NTRAIT,NMG,NPG,NCE,NUE,NDISEQ
0.0 0.0 0.0        << MUT LOCUS, MUT RATE, HAPLOTYPE FREQUENCIES (IF 1)
1 2 3              << locus order
1 2                << affection, No. of alleles
 0.010000 0.990000 << gene frequencies
 3                 << No. of liability classes
0.44 0.44 0.0166
0.44 0.44 0.0166
0.44 0.44 0.0166
3 2                << allele numbers, No. of alleles
 0.67 0.33
3 3                << allele numbers, No. of alleles
 0.32 0.32 0.36
0 0                << SEX DIFFERENCE, INTERFERENCE (IF 1 OR 2)
0.5 0.12           << recombination fraction r[NLOCI-1]
1 0.10000 0.45000  << REC VARIED, INCREMENT, FINISHING VALUE
0.2 0.4  0 0   0 0 << simulated model beta[NTRAIT][NMG+NPG+NCE+NUE]
0.2 0.2            << path coefficient kce[NTRAIT][NCE]
\end{verbatim}\end{small}

Indeed it is very similar to {\bf LINKAGE} parameter file, except extra parameters
are specified in the first line and after the usual {\bf LINKAGE} parameter file
finishes if there are disequilibrium pairs. However the dominance parameter does
need to be specified, with an extra line after the locus type and allele frequency line.
\begin{verbatim}
4 2 << locus type, number of alleles
0.01 0.99 << allele frequencies
0.5 << d, the dominance
\end{verbatim}

The regression coefficients, path coefficients of common environment, and haplotype
frequencies of disequilibrium pairs are specified as separate blocks after a typical
end of LINKAGE parameter file, i.e., after ``variation of recombination''.
For each trait, the regression coefficients are arranged into one line, so are
the path coefficients of common environment transmission.
For a pair of loci in disequilibrium the haplotype frequencies, the ordinal number
of the first locus is also necessary, so the whole information is specified as
follows, Note that the loci are numbered from 0.

\begin{verbatim}
first locus number
the disequibrium matrix
\end{verbatim}
If there are more than one pairs then this is repeated for each pair. It might be
redundant to input the allele frequencies for loci in the pair but it is just a
matter of convenience.

\bigskip\noindent{\bf ped.tst}
\begin{verbatim}
1  1  0  0 2  2 1   1 1 1
1  2  0  0 1  1 1   1 1 1
1  3  2  1 2  2 3   1 1 1
1  4  2  1 1  2 1   1 1 1
1  5  2  1 1  2 1   1 1 1
1  6  2  1 1  2 1   1 1 1
1  7  0  0 1  0 1   0 0 0
1  8  0  0 2  0 1   0 0 0
1  9  7  8 2  1 1   1 1 1
1 10  2  1 2  2 1   1 1 1
1 11  7  8 1  1 1   1 1 1
1 12  6  9 1  1 1   1 1 1
1 13  6  9 1  1 1   1 1 1
1 14  6  9 1  2 1   1 1 1
1 15 11 10 1  1 1   1 1 1
1 16 11 10 1  2 1   1 1 1
... ...
6  1  0  0 1  1 1   0 0 0
6  2  0  0 2  1 1   0 0 0
6  3  0  0 1  1 1   0 0 0
6  4  1  2 2  2 1   1 1 1
6  5  1  2 2  2 1   1 1 1
6  6  1  2 1  1 1   1 1 1
6  7  0  0 2  1 1   0 0 0
6  8  1  2 1  1 1   1 1 1
6  9  1  2 1  1 1   1 1 1
6 10  0  0 2  1 1   1 1 1
6 11  3  4 1  1 1   1 1 1
6 12  3  4 1  2 1   1 1 1
6 13  3  4 1  2 1   1 1 1
6 14  3  4 1  2 1   1 1 1
6 15  3  4 2  1 1   1 1 1
6 16  3  4 1  2 1   1 1 1
6 17  3  4 2  1 1   1 1 1
6 18  6  7 1  2 3   1 1 1
6 19  6  7 2  1 1   0 0 0
6 20  6  7 2  2 2   1 1 1
6 21  9 10 1  2 1   1 1 1
6 22  9 10 1  1 1   0 0 0
6 23  9 10 1  1 1   1 1 1
6 24  9 10 1  1 1   1 1 1
6 25  9 10 2  1 1   1 1 1
6 26  9 10 2  1 1   1 1 1
\end{verbatim}

We see that each line contains pedigree id, individual id, parent ids, sex and marker
availability codes taking value of 1 if the genotype is available or 0 otherwise.
The pedigree file contains the pedigrees in pre-MAKEPED (or post-MAKEPED format, then
\#USEMAKEPED in the header file/sim.h has to be activated. This is only useful when 
there are loops in the family and sex-linked disorders) and with indicators specifying if an individual
has genotype at that locus or not. Note it has the same requirement as in SIMULATE, 
i.e., the order has to be such that parents' ids precede their offsprings.

\bigskip\noindent{\bf problem.dat}
\begin{verbatim}
10         << number of replicates
1232 2122  << random number seeds
\end{verbatim}

This file contains the number of replicates and two random number seeds. Owing to RANLIB,
the eligible ranges for random number seeds are huge, in [1,$\,$ 2,147,483,562] and 
[1,$\,$ 2,147,483,398], repectively.

Two demo files named sim\_loc.tst and sim\_ped.tst show all the features, they can be used
by the command:

\bigskip
SIM sim\_loc.tst sim\_ped.tst
\bigskip

A driver program {\bf diseq} is provided to simulate nuclear families assuming
linkage disequilibrium. The sibship size distribution is based on the popular negative binomial
distribution (Cavalli-sforza and Bodmer 1971) or truncated Poisson distribution.

{\bf Remark} A version of CHRSIM, SIMULATE, together with the LINKAGE utility routine makeped.c
(available from Rockefeller) have been included as separate programs. 
The algorithms in CHRSIM has been extended to be able to include random markers based on 
recent estimates of chromosome lengths and Genethon map (Dib et al 1996), which would make 
it more flexible and suitable for genome scanning research.


\section{Testing of the program}

The first test could involve test of HWE. We can also consider further genetic analysis. 
One possibility is to use POINTER to recover the model for nuclear families, where the
maximum-likelihood estimation was performed by iterating upon the following: $V$,
variance of $x$; $u$, mean of $x$; $d$, degree of dominance at major locus; $q$, gene
frequency at major locus; $t$, displacement at major locus; $H$, polygenic heritability
$(C/V)$; and $B$, relative variance due to common environment $(S/V)$. If only affection
status can be specified, it is defined on $x$, hence $W=0$;  mean and variance of $x$ are
arbitrary and can be taken as $V=1$ and $u=0$. Affection occurs whenever liability $x$ is
greater than some threshold. For such a specification to be independent of any particular
model of inheritance in segregation analysis, one must provide an estimate of the prior
probability of affection in the reference population.

Other programs such as PAP and MORGAN may also be used.

Another example is the ``disequilibrium and causation differentiation".  This is pending 
to be further explored.

\section{Some notes}

The first is about sibship distribution.  As indicated by Morton et al (1983), phenotype
specification may concern a quantitative measurement, or affection. Ideally some
demographic model can be incorporated, as in POPGEN, so age of onset can be considered. 

The distribution of family sizes in the population was
described in Cavalli-Sforza and Bodmer (1971), where they found negative binomial distribution
gives better fit to the data they used. While Ewens (1982) and
Morton (1982) 
showed that family-size distribution should not affect the result of segregation analysis, the
distribution of family sizes can affect power calculations because a data set containing
100 two-child families contains less information than, say 100 five-child families.

The program could be used for provding data in multiple traits analysis.

The program generates genetic data purely from gene dropping through families and does
not consider conditioning of relatives phenotype as in SLINK. It is thus fast but
restricted in terms of analysis.  The program is very flexible in terms of disequilibrium
specifications.

Another program, developed by Kaplan et al for linkage disequilibrium analysis in founder
population, beared the same name.

Would it be worthwhile to convert the SAS optimizer to CFSQP ?

Would it be optimal if size controlled by command-line parameters ? (as in
SPLINK/TRANSMIT, 21-4-1999)

\section{Program history}
\begin{itemize}
\item 6/3/1997 first draft
\item 29/5/1997 changed to get families first, setup for read-ins
\item 6/97 add ad-hoc program used for disequilibrium problem
\item 26/8/1998 improved code and documentation (where is it now ?)
\item 9/3/1999 submitted as attachment with minor change to GAW12
\item 22/4/1999 expanded documentation
\end{itemize}

\section{References}
\begin{enumerate}
\item Boyle, CR and Elston, RC (1979) Multifactorial genetic models for quantitative traits
in humans. Biometrics 35: 55-68.

\item Cavalli-Sforza, LL and Bodmer, WF(1971). "The Genetics of Human Populations. San
Francisco, W.H.Freeman. pp310-313. 

\item Curtis, D. and Sham, PC (1995) Model-free linkage analysis using likelihoods. Am. J. 
Hum. Genet. 57:703-716. 

\item Elston, RC (1980) Segregation analysis. in Current Developments in Anthropological
Genetics, vol I, edited by Mielke, JH, Crawford, MH, New York, Plenum Press, pp327-354.

\item Ewens, WJ (1982) Aspects of parameter estimation in ascertainment sampling schemes
Am. J. Hum. Genet. 34: 853-865.

\item Greenberg, DA (1984) Simulation studies of segregation analysis: application to
two-locus models. Am. J. Hum. Genet. 36:167-176. 

\item Hasstedt, SJ, Meyers, DA, Marsh, DG (1983) Inheritance of immunoglobin E: genetic 
model fitting. Am. J. Med. Genet. 14: 61-66.

\item Khoury, MJ, Beaty,TH and Cohen, BH (1993) Fundamentals of Genetic Epidemiology. Oxford
University Press, Inc. 

\item Lalouel, JM and Morton, NE (1981) Complex segregation analysis with Pointers. Hum. 
Hered. 31: 312-321. 

\item Morton, NE and MacLean, CJ (1974) Analysis of family resemblance. III. Complex
segregation analysis of quantitative traits. Am. J. Hum. Genet. 27:365-84. 

\item Morton, NE(1982) Trials of segregation analysis by deterministic and macro
simulation. Am. J. Hum. Genet. 34:187A.

\item Morton, NE, Rao, DC and Lalouel, JM (1983) Methods in Genetic Epidemiology. Karger. 

\item Press, WH, Teukolsky, SA, Vetterling, WT, and Flannery, BP (1992) Numerical Recipes. 
The Art of Scientific Computing. Second Edition. Cambridge University Press. 
http://cfata2.harvard.edu/nr/.

\item Risch, N(1984) Segregation analysis incorporating linkage markers. I. single-locus models
with an application to Type I diabetes. Am. J. Hum. Genet. 36:363-386. 

\item Schork, NJ (1992) Detection of genetic heterogeneity for complex quantitative
phenotypes.  Genet. Epidemiol. 9: 207-223.

\item Sham, PC(1998) Statistics in Human Genetics. Edward Arnold. 

\item Terwilliger, JD and J Ott (1994) Handbook of Human Genetic Linkage. The Johns Hopkins
University Press.

\item Thompson, EA and Cannings, C(1979) Sampling schemes and ascertainment. In The Genetic
Analysis of Common Diseases: Applications to Predictive Factors in Coronary Heart Disease.
C.F.  Sing and M Skolnick, (eds) Alan Liss, New York. 

\item Thompson, R(1977a) The estimation of heritability with unbalanced data. I. Observations
on parents and offspring. Biometrics 33: 485-495. 

\item Thompson, R(1977b) The estimation of heritability with unbalanced data. II. Data
available on more than two generations. Biometrics 33: 497-504. 

\item Weir, BS(1990) Genetic Data Analysis - Methods for Discrete Population Genetic Data.
Sinauer Associates, Inc. Publishers, Sunderland, Massachusetts.
\end{enumerate}

\section{Contact information}

I would be very pleased to get your views about the program. If you find any problem please 
feel free to contact me at the following address.

\bigskip

Jing Hua Zhao

Department of Psychological Medicine, Institute of Psychiatry, De Crespigny Park, Denmark
Hill, London SE5 8AF, UK. Tel +44 (171) 9193534, Fax +44 (171) 7019044

j.zhao@iop.kcl.ac.uk

\end{document}

\subsection{A note about the disease model}

Assume the population prevalence is known and denoted by $K_p$, we then use a logic similar to
MFLINK to get disease allele frequency. 

Since $$K_p=(1-q)^2f_0+2q(1-q)f_1+q^2f_2$$ we can rewrite it as
$$(f_0-2f_1+f_2)q^2+2(f_1-f_0)q+f_0-K_p=0\quad or\quad aq^2+bq+c=0$$ 
suppose we know $f_0, f_1, f_2$, and $f_0-2f_1+f_2=0$ then $q=\frac{K_p-f_0}{2(f_1-f_0)}$.
In general, by $f_0 \leq f_1 \leq f_2$, $f_0 \leq K_p \leq f_2$, $f_0,f_1,f_2,q \in [0,1]$,
$q$ is $$\frac{-b+\sqrt{b^2-4ac}}{2a}$$ The second root is discarded.

A note about the penetrance model-free linkage analysis. By imposing constraint over $q$ and
$F=(f_0,f_1,f_2)$ by $K_p$, $q$ and $F$ could only take values in a polyhedron.  As it may
involve multidimensional optimization and computationally demanding.  To avoid this, we only
evaluate likelihoods for transmission models represented by points on the lines joining
(0,0,1) to $(K_p,K_p,K_p)$ and then to (0,1,1)  which imply considering only transmission
models between pure recessive and no single-locus effect, then to pure dominant.  In this
situation, if we vary $f_1$ from 0 to 1, both $f_0$ and $f_2$ are functions of $f_1$:  if
$f_1<K_p$, $f_0=f_1,f_2=\frac{f_1(K_p-1)}{K_p+1}$, otherwise $f_2=f_1$,
$f_0=\frac{(1-f_1)K_p}{1-K_p}$. 

\subsection{An example -- disequilibrium and causation differentiation}

At this moment the program keeps the original example as diseq.c, i.e.,
differentiation of disequilibrium and causation.  Consider two biallelic markers with linkage
disequilibrium parameter $d$, the joint and marginal frequencies can be shown in the following
$2 \times 2$ table:

\begin{center}
\begin{tabular}{cc|c}
$p_1q_1+d$ & $p_1q_2-d$ & $p_1$ \\
$p_2q_1-d$ & $p_2q_2+d$ & $p_2$ \\ \hline
$q_1$ & $q_2$ & 1 \\
\end{tabular}
\end{center}

To ensure all four cells positive, we have $\max(-p_1q_1,-p_2q_2)=-\min(p_1q_1,p_2q_2)\leq
d\leq \min(p_1q_2,p_2q_1)$. From constraints for the marginal probabilities $p_1q_1+d\leq p_1$
and $p_2q_2+d\leq q_2$ we have $d\leq \min(p_2q_1,p_1q_2)$. In summary,
-$\min(p_1q_1,p_2q_2)\leq d \leq \min(p_1q_2,p_2q_1)$. 

Now suppose we have another biallelic marker in equilibrium with the second marker, the
haplotype frequencies would then be as follows,

\begin{verbatim}
p2=1-p1;
q2=1-q1;
r2=1-r1;
x1=(p1*q1+d)*r1;
x2=(p1*q1+d)*r2;
x3=(p1*q2-d)*r1;
x4=(p1*q2-d)*r2;
x5=(p2*q1-d)*r1;
x6=(p2*q1-d)*r2;
x7=(p2*q2+d)*r1;
x8=(p2*q2+d)*r2;
\end{verbatim}

Meticulous care should be taken to make sure these haplotype frequencies are properly
specified. 

So far as SAS/IML is concerned we need to redefine our problem, with respect to the
constraint $(1-q)^2f_0+2(1-q)qf_1+q^2f_2=K_p$.

The smallest $f_1$ is when $f_0=f_1, f_2=1$, $f_1=\frac{K_p-q^2}{1-q^2}$, $f_1\ge
\max(\frac{K_p-q^2}{1-q^2},0)$. The biggest $f_1$ is when $f_0=0,f_1=f_2$,
$f_1=\frac{K_p}{1-(1-q)^2}$,
$f_1\le\min(\frac{K_p}{1-(1-q)^2},1)$.

Given $f_1,q$, if $f_1<K_p$, the smallest $f_2$ is when $f_0=f_1$, i.e.
$\frac{K_p-(1-q^2)f_1}{q^2}$;
If $f_1>k$, the smallest $f_2=f_1$. The largest $f_2$ is when $f_0=0$, i.e.
$\min(\frac{K_p-2q(1-q)f_1}{q^2},1)$.

Given $q,f_1,f_2$, $f_0=\frac{K_p-2q(1-q)f_1-q^2f_2}{(1-q)^2}$.

Given the lower bound $L$ and upper bound $U$ of each parameter, then the problem is converted
into a nonlinear optimization with boundary constraints. $x=L+c(U-L),$ where c varies between
0 and 1.  Further conversion using logistic function $f(x)=\frac{1}{1+\exp(-x)}$ removing the
boundary constraints is possible. 

Finally comes to the test of closely-linked and disequilibrium or the disease itself. In
the first instance we just proceeds like before while in the second instance $d$ has to be
$pq$, thus no estimation for $d$ is necessary. 

\begin{tabular}{cc|c}
$p^2+d$ & $pq-d$ & $p$ \\
$pq-d$ & $q^2+d$ & $q$ \\ \hline
$p$ & $q$ & 1 \\
\end{tabular}
, when $d=pq$ it becomes
\begin{tabular}{cc|c}
$p$ &  0  & $p$ \\
0   & $q$ & $q$ \\ \hline
$p$ & $q$ & 1 \\
\end{tabular}

This leads to a test of 2 degree of freedom between the two.

Suppose we generate data from the second instance but try to analyze it using the first
instance, then we are thinking the first marker is the disease locus, so is its allele
frequency. 

If we suppose to search for $K_p$ compatible sets of penetrance and frequency allowing for
linkage disequilibrium, then we could vary $q$ between 0 and 1 and $d$ within $(-pq,pq)$. 
Denote the marker locus by capital letter B, the associated haplotype frequencies by
$h_{11},...,h_{1n}, h_{21},...,h_{2n}$, $n$ is number of alleles. Conditional on
individual's affection status, the genotypes are then obtained as follows (see, Sham
1998). 

\begin{eqnarray}
P(B_i B_i | A) &=& \frac{f_0 h_{1i}^2+f_1 (2h_{1i}h_{2i})+f_2
h_{2i}^2} {K_p} \cr
P(B_i B_j | A) &=& \frac{f_0 (2h_{1i}h_{1j})+f_1 (2h_{1i}h_{2j}
+2h_{1j}h_{2i})+f_2 (2h_{2i} h_{2j})} {K_p} \cr
P(B_i B_i | U) &=& \frac{s_0 h_{1i}^2+s_1 (2h_{1i}h_{2i})+s_2
h_{2i}^2} {Q} \cr
P(B_i B_j | U) &=& \frac{s_0(2h_{1i}h_{1j})+s_1(2h_{1i}h_{2j}+2h_{1j}h_{2i})+s_2(2h_{2i}h_{2j})}
{Q}
\end{eqnarray}

Where $Q=1-Kp$, $s_0=1-f_0$, $s_1=1-f_1$, $s_2=1-f_2$. Assume the same genotype distribution, we
can explore the behavior of disease-marker association.

In simulation of the problem, a simple selection, i.e., for any families with more than
two affected, was adopted. 

